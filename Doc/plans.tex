\chapter{Plans for the future}

\section{Minor issues}

There are tons of minor issues to fix.  Here is a partial list.

\subsection{Notation issues}

\begin{itemize}
\item Elements now have an additional property that lets the user
  position them horizontally with an offset relative to their normal
  aligned position, so that notes with an interval of a second form
  different layers do not overlap physically.  But we also have to
  introduce a set of primitives to alter this offset, and perhaps some
  visual indications that show the offset.  Perhaps we should show a
  thin yellow line where the alignment is whenever there is a nonzero
  offset.  I am suggesting yellow, because it would not destroy the
  overall visual impression of the score. 
\item The cursor is currently drawn by the drawing routine for the
  layer, but that routine is not supposed to know where the cursor
  is.  We need to separate out cursor drawing from layer drawing. 
\end{itemize}

\subsection{Implementation details}

\begin{itemize}
\item Playing as MIDI should create a temporary file with a unique
  name in \texttt{/tmp} as opposed a file in the current directory
  with a name that can clash with others. 
\item Someone who knows how pathnames work needs to find out how to
  use them in place of strings for filenames. 
\end{itemize}

\section{Major issues}

\subsection{Implementation details}

\begin{itemize}
\item Different information might need to be attached conceptually to
  barlines, in particular: fermata, repeat bars, etc.  It would be
  natural to physically attach this information to the bar, indicating
  properties of the barline at the end of the bar, but there is a
  conceptual problem with this, since some of this information is not
  unique to the layer, but sometimes global to all layers that are
  traversed by the resulting barline and sometimes even (repeat bars?)
  to the entire score.  One could imagine a separate structure for the
  segment, indicating positions of repeat bars and such.  It would
  have the additional advantage that if a barline is deleted in a
  layer which has a repeat bar further on, the visual effect would be
  nicer.  Operating on this new structure should be possible from any
  layer. 
\item The fermata symbol is sometimes centered on a beam (tremolo).
  We currently have no way of attaching any information to a beam,  
\end{itemize}

\section{Minor projects}

\subsection{Default staff line for rests}

When using several voices (usually two) on one staff line, one would
like for rests to be placed on (say) the upper staff line for the
upper voice and (say) the lower staff line for the lower voice rather
than, as now, having all rests inserted by default on the middle staff
line.  It would be nice if a layer had a default staff line for rests
so that changing hte layer automatically changes the default staff
line for rests.  

\subsection{Default stem direction for a layer}

When using several voices on a staff line, one somtimes wants the
stems of the upper voice to be up and the ones of the lower voice to
be down.  It would be nice for a layer to have a default stem
direction so that changing the voice would automatically change the
stem direction.  

Here is a possible specification of such a feature:

A new element is still created according to the stem direction of the
input state without taking into account the stem direction of the
layer.  When a layer is displayed, the following final stem direction
is computed:

\begin{itemize}
\item if the stem direction of the element is not \emph{auto} then
  that stem direction is used;
\item if the stem direction of the element is \emph{auto}, but the
  default stem direction of the layer is not \emph{auto}, then the
  default stem direction of the layer is used;
\item otherwise, the stem direction is computed from the position of
  the notes of the element. 
\end{itemize}

\section{Major projects}

\subsection{Menu items with arguments}

This is a {\clim} project.  Currently {\clim} does not know how to
read arguments when a command has been invoked from a menu item.
Consequently, we are restricted to menu items that have no arguments. 

\subsection{Layout by page}

We need to have a more sophisticated layout algorithm that divides the
score into pages rather than just lines.  One interesting idea seems
to be to use the {\obseq} library in a nested kind of way, where an
invocation of the cost combination method on the global level would
provoke another call to the \texttt{solve} function on the page
level.  This would be really cute, actually, and it would show the
power of the approach of the {\obseq} library.  

Here is how it would be done.  First it requires a minor modification
to the {\obseq} library.  Instead of the elements \emph{inheriting}
from \texttt{obseq-elem}, they need to be separate so that an element
can be contained in more than one \texttt{obseq}.  This modification
has some consequences: the operations to compute the next and previous
element need to take an \texttt{obseq} argument to dispatch on.  The
operations \texttt{last-undamaged-element} and
\texttt{first-undamaged-element} must be modified and renamed to (say)
\texttt{tail-valid-elements} and \texttt{head-valid-elements} and take
a number instead of an element, since the \texttt{obseq} itself does
not have a mapping from its internal elements to the client element.
This requires the client to know how many elements there are, which is
a minor inconvenience (but still an inconvenience). Then the seq-cost
on the global level will be an obseq on the page level, and a call to
combine a seq-cost and an element on the global level would trigger a
call to solve on the page level.

We also need to have different cost combination strategies, for
instance:

\begin{itemize}
\item Paper roll with last line not right justified.  This involves
  assigning a low cost to the last line even when it has to be highly
  stretched in order to fill the line.  
\item Paper roll with last line right justified.  This is all we have
  right now.
\item Pages with last page not filled and last line on last page not
  filled. 
\item Pages with last page filled normally.
\item Pages with a fixed number of filled pages.
\item Pages with an arbitrary number of pages, but the number must be
  of the form $n = ax + b$.
\end{itemize}
