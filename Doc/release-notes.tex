\chapter{Release notes}

\section{Release 0.3 (2003-??-??)}

\subsection{Features added from 0.2}

\begin{itemize}
\item An Emacs-style keyboard macro facility has been added. 
\item Right edge of staff is no longer hard wired.
\item Left margin is now a parameter of the buffer (and on disk).
\item x-position of first measure on line now depends on the size of
  the key signature.  This means that wide key signatures no longer
  overlap with notes of the first measure on the line.
\item Improved MIDI generation making it no longer necessary to pad 
measures with empty clusters.
\item Fixed a problem that made {\gs} crash on new versions of
  McCLIM (thanks to Andy Hefner for figuring this out). 
\item {\gs} can now use asdf instead of mk:defystem (thanks to
  Christophe Rhodes).
\item {\gs} now runs on SBCL as well as on CMUCL (thanks to Christophe
  Rhodes).
\item Fixed problem with flickering introduced by new version of McCLIM.
\item More menu commands that need arguments now work correctly.
\item Staves are now clickable.  Commands that take staves as
  arguments now prompt for existing staves as opposed to just names of
  staves. 
\item Added completion for clef types and staff types. 
\end{itemize}

\subsection{Bug fixes from 0.2}

\begin{itemize}
\item fixed a bug in pixmap cacheing
\item fixed pathname-related problems for MIDI and fonts. 
\item fixed annoying bug that ruined the spacing of unevenly timed
  music. 
\item last sharp sign in key signature with seven sharp signs is now
  drawn. 
\item fixed conformance bugs in some \texttt{loop} constructs (thanks to
  Christophe Rhodes).
\end{itemize}

\section{Release 0.2 (2003-09-06)}

\subsection{Features added from 0.1}

\begin{itemize}
\item The command for inserting a bar line (\kbd{|}) now works
  correctlhy in that it insert the bar line where the cursor is and
  not right before the next bar line.
\item Erasing and deleting a bar line now works as announced.
\item Rests get inserted on the current staff line. 
\item It is now possible to move rests vertically on their staff
  lines. 
\item Multiple staves.
\item Key signatures.
\item A new note inserted inherits the accidentals of the staff line
  it is initially placed on.  This avoid for the user to have to
  constantly add accidentals to newly inserted notes in key signatures
  with many sharps or flats.
\item Reading and writing {\gs} files is better organized so that it
  will be easier to bump the file version while maintaining backward
  compatibility.  I do not intend to maintain the ability to write
  older version files, which unfortunately makes {\gs} behave like
  commercial products with planned obsolescence, the difference being
  that the upgrade of {\gs} can be had for free.
\item Much improved cursor display.  The cursor now reflects the
  current staff as well as the interval used to add new notes.
\item Double buffering of output to avoid flickering.
\item The reference manual now has a list of commands with associated
  keys. 
\end{itemize}

\subsection{Bug fixes from 0.1}
\begin{itemize}
\item Creating a new buffer did not erase the old input state, in
  particular the staff of the old input state. 
\item There was a bug in the beaming algorithm that created stems of
  abnormal length. 
\item Fixed a bug in the computation of the dominating note of a
  cluster that made stems to short in beamed multi-note clusters with
  stems down.
\item Empty clusters are now displayed correctly (i.e., not at all)
  without signaling a condition. 
\end{itemize}

\section{Release 0.1 (2003-08-24)}

\subsection{What works}

\subsubsection{Basic music notation}

\begin{itemize}
\item Noteheads
\item Stems with the right dimensions and the right position
  calculated automatically, 
\item Augmentation dots, though placed a bit too close to the
  notehead. 
\item A single main beam with the correct placement and slant for all
  cases. 
\item Correct placement of all noteheads and accidentals in
  arbitrarily complicated clusters. 
\item Automatic generation of as many ledger lines as called for.
\item Barlines, though currently drawn as lines rather than as
  rectangles. 
\item Rests.
\end{itemize}

\subsubsection{Implementation issues}

\begin{itemize}
\item The {\obseq} library is in pretty good condition.
\item The (fairly complicated) beaming algorithm for a single beam is very satisfactory.
\item The (fairly simple) algorithm for determining when a note should
  be suspended seems to work. 
\item The algorithm for placing accidentals in complicated clusters is
  also very satisfactory.
\end{itemize}

\subsubsection{User interface}

\begin{itemize}
\item Most of the basic user interface is in place.
\end{itemize}

\subsection{What does not work}

\subsubsection{Basic music notation}

As the reader can tell, most of the basic music notation is still
missing.  Here is a (probably incomplete) list from Ross:

\begin{itemize}
\item Augmentation dots in the presence of suspended notes 
\item Multiple beams.  This is not particularly hard to put in.  
\item Multiple staves.
\item Staves do not have time signatures yet.  This is a problem,
  because the file format must be changed to include that.
\item Most accents. 
\item Slurs and ties.  I don't even know how to do it at this point. 
\item Repeat bars
\item Key changes in the middle of a slice. 
\item Brackets and braces.
\item Some minor spacing decisions are wrong (beginning of the line to
  clef, clef to first note, etc)
\item Triplets and generalizations thereof.
\item Lyrics
\item Fermata
\item Dynamics
\item Cue notes and grace notes
\item Passing tones
\item Ending signs
\item Fingering
\item Trill
\item Glissando
\item Arpeggio signs
\item Tremolo
\item Bowing and stroking marks
\item Pedal marks
\item Repeat signs
\item Octaves
\item Pause
\end{itemize}

\subsubsection{Implementation issues}

\begin{itemize}
\item Reorganizing output records to substitute different glyphs. 
\item Need to implement the Lime algorithm for distributing space (or
  lack thereof) on a line.  Currently clusters with complicated stacks
  of accidentals will overlap preceding clusters. 
\item We need to present a consistent interface to the combination of 
  buffer module and the cursor module, and then write all primitives
  in terms of that interface.  This would be the interface that is
  presented to the extension writer.  Using this interface, it should
  not be possible to break any of the invariants of the other
  interfaces, such as numbering of all entities and maintaining
  modified flags for redisplay.  In particular, this interface does
  not contain the concept of a 'bar', which should remain internal to
  the buffer module.  
\end{itemize}

\subsubsection{User interface}

\begin{itemize}
\item It would be nice to mark the current note in the score pane as
  opposed to in a separate pane.  Idea: draw a yellow square
  background under the current note.  It may be possible to get rid of
  the current element pane altogether. 
\end{itemize}

